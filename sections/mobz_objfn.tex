Up to this point, we have focused on the commuting zone methodology exclusively. However, in the next two sections we turn to two important issues: how to measure labor market integration, and an alternative clustering method that does not have the drawbacks of the hierarchical technique.

The theoretical literature consistently defines a local labor market as characterized by similar wages, unemployment rates, and commuting links within the area. However, there are no established methods for measuring how appropriate a set of local labor markets is in approximating the theoretical area.

To that end, we formulate an objective function to evaluate how well any given local labor market definition reflects this theoretical construct. To be specific, we measure four statistics that reflect the integration of an area in terms of unemployment and wage series as well as in- and out- commuting rates.\footnote{We have considered additional measures that correspond to the theoretical literature, including housing or rental price series and compactness, which could be measured by average pairwise distance.} 

First, for both unemployment rate and wages (using BLS data), we measure the average pairwise correlations between counties in a cluster, which are $\bar{\rho}_i^{URATE}$ and $\bar{\rho}_i^{Wages}$, respectively.\footnote{The pair-wise correlations between counties are calculated using six years of data (for this paper, that is 1990-1995).
\[
\bar{\rho}_i^{URATE} = \frac{1}{2N} \sum_{i \in C} \sum_{j \in C} \rho_{i,j}
\]} The higher these values are, the more integrated the counties are.

Second, we measure commuting flows into the labor market from other counties, as a share of the local labor force, as well as commuting flows to outside areas ($InflowShare_i$ and $OutflowShare_i$). We expect that the commuting flows within a labor market ought to be much higher than commuting flows outside of the labor market, reflecting an integrated area, such that a higher value of these measure reflects lower labor market integration. We sum these measures across labor markets in the following manner:

\begin{equation}\label{eqn:objfn}
	Objfn(C) = \frac{1}{4N_c} \sum_{i\in C} (\gamma_1 \bar{\rho}^{URate}_i + \gamma_2 \bar{\rho}^{Wages}_i - \gamma_3 InflowShare_i - \gamma_4 OutflowShare_i )
\end{equation}

Where $Objfn(C)$ is the objective function value for a definition of local labor markets, $C$ The $\gamma$ factors are normalization weights, since $\bar{\rho}$ and the commuting shares are on different scales.\footnote{We normalize these values by taking random clusters of 5 counties and calculating each component of equation \ref{eqn:objfn}, then calculating the mean; in this exercise, the values of $\gamma$ are just the inverse of these means, although there may be other candidate weights.} Given that a larger objective function value reflects a more integrated local labor market, when comparing two competing definitions of local labor markets, the definition with the higher objective function value is the better labor market definition. More formally, for two local labor market definitions, $C$ and $D$, if $Objfn(C)>Objfn(D)$, then $C$ is the more appropriate local labor market.

In the next section, we develop a methodology using this objective function, such that our resulting local labor market definitions maximizes the objective function.

%\subsection{Benchmarking Objective Function Values}

%To have a benchmark to normalize objective function values, so that they are comparable over time, we propose using a ``worst-case'' local labor market. Specifically, we randomly group counties into sets of 5, irrespective of where they are in the country. These definitions should not reflect any labor market integration other than the underlying integration of the national labor market, which varies over time, and allows us to take out changes in the objective function over time based on overall dispersion of labor market conditions or time-varying quality of the available data.\footnote{In practice, this exercise is similar to adding year fixed-effects to a regression, in order to absorb variation that is common to the country as a whole in a given year. Because dispersion in unemployment and wages rates is different across years and business cycles nationally, some of the changes in $Objfn(C)$ reflect national-level phenomena rather than local labor market integration.}

%% table cluster changes
\begin{table}\centering
\caption{Summary Statistics of the Allocation of Counties to Clusters, by Local Labor Market Definition\label{tab:clustersum}}
\begin{tabular}{lccccccc}
\hline\hline
Cluster scheme & Clusters & Counties & Median & Mean & StdDev & Min & Max \\ 
\hline
Commuting Zones 1990 & 741 & 3,141 & 4 & 4.24 & 2.50 & 1 & 19 \\ 
Commuting Zones 2000 & 709 & 3,141 & 4 & 4.43 & 2.48 & 1 & 19 \\ 
CBSAs and State remainders & 962 &    3,142 &  1 & 3.2661 &  8.1104 &   1 &    126 \\ 
States & 51 & 3,142 & 62 & 61.61 & 46.76 & 1 & 254 \\ 
Randomized Zones & 621 & 3,105 & 5 & 5 & 0 & 5 & 5 \\ 
Mobility Zones & 512 & 3,108 & 6 & 6.07 & 2.34 & 1 & 14 \\ 
\hline
\multicolumn{8}{p{6.5in}}{\footnotesize \textit{Notes}: The definitions of 1990 and 2000 Commuting Zones are based on the TS1996 methodology and are the clusters provided by the Economic Research Service. CBSA mappings of counties are based on Office of Management and Budget definitions. Randomized Zones and Mobility Zones are Authors' calculations, with Mobility Zones using public-use tabulations of Journey to Work for the 1990 Census as well as the Local Area Unemployment Statistics from the Bureau of Labor Statistics and earnings data from Bureau of Economic Affairs.}\\
\end{tabular}
\end{table}



%These results are striking. We see that in 1990, the 1990 Commuting Zones have the lowest objective function values, while states and CBSAs have much higher values. But in 2000, the 1990 commuting zones are less appropriate, and instead 2000 commuting zones better reflect local labor markets; this difference persists in 2010. These results highlight the fact that over time, fixed local labor market definitions become less reflective of the true local labor markets. As the nation as a whole becomes more integrated, states do better at reflecting local labor markets, but still much worse than the commuting zones. \mc{}{MK: any explanation for the drop in all measures from the 1990 data to the 2000 data? If the country is more integrated, detailed area measures should be worse off. Also, would it be worth producing a measure where each cluster consists of just one county, as the extreme of disaggregation? }

%{\footnotesize \textit{Notes}: The cluster mappings cz1990 and cz2000 refer to the definitions of 1990 and 2000 Commuting Zones. State and MSA refer to mappings of counties to States and CBSAs. Fastclus refers to the Mobility Zones mapping, based on the k-means methodology known in SAS as PROC FASTCLUS. All objective function evaluations are normalized by the evaluation of Random Zones.}

%In the next section we outline an alternative methodology for defining local labor markets; we show the outcomes of this analysis in Figure \ref{fig:objfn_norm_fastclus}, comparing it with the 1990 Commuting Zones.
