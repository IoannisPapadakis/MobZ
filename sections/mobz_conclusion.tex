Numerous influential papers in labor economics have used commuting zones as an alternative definition to local labor markets. However, researchers typically do not evaluate how the methodology used to construct commuting zones may impact their findings, nor have there been any evaluations of the sensitivity of commuting zones to design feature more generally. Our paper contributes to this literature by analyzing this methodology and its implications for empirical applications.

We document that the commuting zone methodology is sensitive to uncertainty in the input data and parameter choices and we demonstrate how these features affect the resulting labor market definitions. Furthermore, we demonstrate that uncertainty in local labor market definitions also affects empirical estimates that use commuting zones as a unit of analysis. Future work may explore other clustering methods, which are less history-dependent, as they may come to more globally optimal solutions. Developing metrics to compare candidate zones against one another will facilitate comparisons of the overlap of different clustering outcomes. Additional metrics of local labor market integration may help to evaluate the overall validity of various definitions.

