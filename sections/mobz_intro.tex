Local labor markets are an important unit of analysis in labor economics. Theoretical papers emphasize characteristics of a local labor market including common wage and rent levels \citep{Roback1982,Moretti2011} as well as job-finding and unemployment rates \citep{HL2012,SS2014} and often assume fixed or variable costs for transferring jobs or workers between labor markets. In empirical labor economics, researchers interested in estimating the effect of some local, exogenous shock on labor market outcomes must decide how to define the set of affected jobs or workers. Researchers examining labor markets in the United States often turn to one of several standard geographic definitions that are widely known and compatible with publicly available economic data, including: states \citep{BK1992,Wozniak2010,KW2011}, metropolitan areas \citep{BH2000,Card2001,Notowidigdo2011,Diamond2016}, and counties \citep{MRR2015,FGS2015}.

Another labor market definition with advantages for some research topics over the above definitions is commuting zones, which are composed of counties and were originally defined by \citet{TS1996} (henceforth, TS). Commuting zones are similar to metropolitan areas in that they are meant to capture areas of economic integration that do not necessarily conform to regional political boundaries (such as states) \citep{FR2000,FR2010}. Unlike metropolitan areas, commuting zones have no urbanized area size requirements and span the entire United States, allowing researchers to measure effects for the entire country rather than just the set of metropolitan areas (or the complements of metropolitan areas within a state). 

Given these features, commuting zones have been used in a number of influential papers in the labor economics literature, including \citet{ADH2013}, as well as \citet{ChettyHendrenKlineSaez2014}, \citet{AM2015}, \citet{Restrepo2015}, and \citet{Yagan2016}. Despite their widespread use, to the best of our knowledge, the methodology underlying commuting zone definitions and its impact on empirical estimates has not received much scrutiny and many researchers do not consider how findings may be sensitive to design issues. 

Our paper makes two contributions for empirical analysis using commuting zones. First, we describe two methodological issues that researchers should be aware of when they use the commuting zone definitions. Second, we show how these methodological issues impact empirical estimates. 

Our findings suggest that researchers should consider evaluating the sensitivity of their results, and we propose two ways that researchers can test if their results are robust to the uncertainty inherent in this definition of local labor markets. These findings are informative to the use of commuting zones for defining local labor markets specifically, but also suggest caution for researchers in general when measuring treatment effects in geographically distinct areas where treatment may not be as discretely related to geography as is implied by the unit of measure. 

The remainder of the paper proceeds as follows. We describe the data we use and the commuting zone definitions and methodology in detail in Section \ref{sec:method}. In Section \ref{sec:dsens} we outline the extent to which commuting zone definitions are sensitive to data inputs and design decisions, and in Section \ref{sec:esens} we discuss how those issues affect empirical estimates and provides guidance for researchers in light of our results. Section \ref{sec:conclusion} concludes and discusses next steps. 


