Local labor markets are an important unit of analysis in labor economics. Theoretical papers emphasize characteristics of a local labor market including common wage and rent levels \citep{Roback1982,Moretti2011}, as well as job-finding and unemployment rates \citep{HL2012,SS2014}. In empirical labor economics, researchers interested in estimating the effect of some local, exogenous shock on labor market outcomes must decide how to define the area which is affected by the shock, and researchers have a number of different definitions of local labor markets from which to choose. \citet{BK1992}, \citet{Wozniak2010}, and \citet{KW2011} use the state, while other researchers use metropolitan areas \citep{BH2000,Card2001,Notowidigdo2011,Diamond2016}, while still others use counties \citep{MRR2015,FGS2015}.

One such labor market definition is commuting zones, which were originally defined by \citet{TS1996} (henceforth, TS1996). Commuting zones have two distinct advantages over the above definitions. First, they span the entire United States, allowing researchers to measure effects for the entire country rather than just metropolitan areas. Second, commuting zones group together counties based on commuting flows, which implies some level of economic integration.\footnote{Metropolitan areas are also based on commuting flows, and are updated regularly to reflect changes in commuting patterns.} The definition acknowledges that labor markets are not constrained by county and state lines, but are based on relevant linkages between counties.

Given these advantages, commuting zones have been used in a number of influential papers in the labor economics literature, including \citet{ADH2013}, as well as \citet{Yagan2016}, \citet{Restrepo2015} and \citet{AM2015}. Despite their widespread use, to the best of our knowledge, the methodology underlying commuting zone definitions and its impact on empirical estimates has not received much scrutiny. 

Our paper makes two contributions for empirical analysis using commuting zones. First, we outline a number of methodological issues that researchers should be aware of when they use the commuting zone definitions. Second, we show how these methodological issues impact empirical estimates. Our findings suggest that researchers should consider evaluating the sensitivity of their results, and we propose two ways that researchers can test if their results are robust to the uncertainty induced by commuting zones. 

The remainder of the paper proceeds as follows. We describe the data we use and the commuting zone definitions and methodology in detail in Section \ref{sec:method}. In Section \ref{sec:dsens} we outline the extent to which commuting zone definitions are sensitive to data inputs and researcher decisions, and in Section \ref{sec:esens} we discuss how those issues affect empirical estimates and provides guidance for researchers in light of our results. Section \ref{sec:conclusion} concludes and discusses next steps. 


