Local labor markets are an important unit of analysis in labor economics, both in the theoretical and empirical literatures. Theoretical papers emphasize characteristics of a local labor market including common wage and rent levels \citep{Roback1982,Moretti2011}, as well as job-finding and unemployment rates \citep{HL2012,SS2014}.

In empirical labor economics, researchers may be interested in estimating the effect of some local, exogenous shock on labor market outcomes, and an important decision in any research design is defining the area that is directly affected by the shock. In estimating this effect, researchers have a number of different definitions of local labor markets from which to choose. \citet{BK1992}, \citet{Wozniak2010}, and \citet{KW2011} use the state, while other researchers use metropolitan areas \citep{BH2000,Card2001,Notowidigdo2011,Diamond2016}, while still others use counties \citep{MRR2015,FGS2015}.

One alternative labor market definition is commuting zones, which were originally defined by \citet{TS1996} (henceforth, TS1996). Commuting zones have two distinct advantages over the above definitions. First, they span the entire United States, allowing researchers to measure effects for the entire country rather than just metropolitan areas. Second, commuting zones group together counties based on commuting flows, which implies some level of economic integration (metropolitan areas are also based on commuting flows). The definition acknowledges that labor markets are not constrained by political boundaries such as county and state lines, but are based on relevant linkages between counties. 

%While the strict assignment of counties to one commuting zone or another overstates the distinctiveness of labor markets, this simplifying assumption facilitates many empirical applications. 

Given these advantages, commuting zones have been used in a number of influential papers in the labor economics literature, including \citet{ADH2013}, as well as \citet{ChettyHendrenKlineSaez2014}, \citet{AM2015}, \citet{Restrepo2015}, and \citet{Yagan2016}. Despite their widespread use, to the best of our knowledge, the methodology underlying commuting zone definitions and its impact on empirical estimates has not received much scrutiny. 

Our paper attempts to fill this gap in the literature by making two important contributions to our understanding of the empirical applications of commuting zone methodology. First, we demonstrate that the methodology used in TS1996 is sensitive to errors in the underlying input data, which overstates the certainty in these labor market areas. Second, we also show that the resulting commuting zones are sensitive to researcher choice. We show how these two aspects of commuting zones affect empirical estimates by replicating Autor, Dorn and Hanson's influential paper.

The remainder of the paper proceeds as follows. In Section \ref{sec:method}, we describe the method that Tolbert and Sizer used to develop commuting zones and attempt to replicate their zones. In Section \ref{sec:dsens}, we evaluate how sensitive this methodology is to errors in the underlying commuting flows data, and in Section \ref{sec:esens}, we replicate empirical estimates that make use of Commuting Zones and evaluate how this sensitivity affects empirical estimates. We conclude in Section \ref{sec:conclusion} by providing recommendations for empirical research.
