% table cluster changes
\begin{table}\centering
\caption{Summary Statistics of the Allocation of Counties to Clusters, by Local Labor Market Definition\label{tab:clustersum}}
\begin{tabular}{lccccccc}
\hline\hline
Cluster scheme & Clusters & Counties & Median & Mean & StdDev & Min & Max \\ 
\hline
Commuting Zones 1990 & 741 & 3,141 & 4 & 4.24 & 2.50 & 1 & 19 \\ 
Commuting Zones 2000 & 709 & 3,141 & 4 & 4.43 & 2.48 & 1 & 19 \\ 
CBSAs and State remainders & 962 &    3,142 &  1 & 3.2661 &  8.1104 &   1 &    126 \\ 
States & 51 & 3,142 & 62 & 61.61 & 46.76 & 1 & 254 \\ 
Randomized Zones & 621 & 3,105 & 5 & 5 & 0 & 5 & 5 \\ 
Mobility Zones & 512 & 3,108 & 6 & 6.07 & 2.34 & 1 & 14 \\ 
\hline
\multicolumn{8}{p{6.5in}}{\footnotesize \textit{Notes}: The definitions of 1990 and 2000 Commuting Zones are based on the TS1996 methodology and are the clusters provided by the Economic Research Service. CBSA mappings of counties are based on Office of Management and Budget definitions. Randomized Zones and Mobility Zones are Authors' calculations, with Mobility Zones using public-use tabulations of Journey to Work for the 1990 Census as well as the Local Area Unemployment Statistics from the Bureau of Labor Statistics and earnings data from Bureau of Economic Affairs.}\\
\end{tabular}
\end{table}

